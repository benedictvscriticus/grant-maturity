\documentclass[conference]{IEEEtran}
\IEEEoverridecommandlockouts
% The preceding line is only needed to identify funding in the first footnote. If that is unneeded, please comment it out.
%Template version as of 6/27/2024

%\usepackage{cite}
\usepackage{amsmath,amssymb,amsfonts}
\usepackage{algorithmic}
\usepackage{graphicx}
\usepackage{textcomp}
\usepackage{xcolor}
\def\BibTeX{{\rm B\kern-.05em{\sc i\kern-.025em b}\kern-.08em
    T\kern-.1667em\lower.7ex\hbox{E}\kern-.125emX}}

\definecolor{dg}{RGB}{64,64,64}

\usepackage[colorlinks = true,
		citecolor = black,
		urlcolor= blue,
		linkcolor=dg,]{hyperref}

\begin{document}

\title{Evaluating Progress in Web3 Grants: Introducing the Grant Maturity Index.
\thanks{This work is has been funded by the EU in the framework of the NGI TRUSTCHAIN project under the European Commission HORIZON-CL4-2022-HUMAN-01-03 grant: \href{https://cordis.europa.eu/project/id/101093274}{101093274} and by \href{https://ens.domains/}{ENS Domains} through its \textit{Large Grants} program.}
}
\author{\IEEEauthorblockN{1\textsuperscript{st} Ben Biedermann}
\IEEEauthorblockA{\textit{Islands and Small States Institute} \\
\textit{University of Malta}\\
Msida, Malta \\
\href{https://orcid.org/0000-0003-1331-6517}{0000-0003-1331-6517}}

\and
\IEEEauthorblockN{2\textsuperscript{nd} Fahima Gibrel}
\IEEEauthorblockA{\textit{Grant Innovation Lab} \\
\textit{Metagovernance Project Inc}\\
Brookline (MA), United States \\
\href{mailto:mgibrel.fahima@gmail.com}{gibrel.fahima@gmail.com}}

\and
\IEEEauthorblockN{3\textsuperscript{rd} Victoria Kozlova}
\IEEEauthorblockA{\textit{WID3\textsuperscript{+} Consortium} \\
\textit{ACURRAENT UG} \\
Frankfurt (Oder), Germany \\
\href{https://orcid.org/0000-0003-3303-3143}{0000-0003-3303-3143}}
}
\maketitle

\begin{abstract}
This report introduces the Grant Maturity Framework (GMF), a novel evaluative framework designed to assess the maturity and operational effectiveness of Web3 grant programs. As Web3 continues to develop, the decentralised nature of these programs brings both opportunities and challenges, particularly when it comes to governance, transparency, and community engagement. Traditional funding models are often governed by standardised processes, but Web3 grants lack such consistency, making it difficult for grant operators to measure the long-term success of their programs.

The \textit{Grant Maturity Framework (GMF)} was created through \textit{exploratory applied research} to address this gap. Inspired by the World Bank’s GovTech Maturity Index (GTMI), the GMF is tailored specifically for the decentralised Web3 ecosystem. The GMI evaluates key dimensions of grant programs—governance, transparency, operational efficiency, and community engagement -- providing grant operators with a clear benchmark for assessing and improving their programs.

The primary objectives of this research are to:
\begin{itemize}
    \item Identify the structural indicators that adequately describe Web3 grant programs.
    \item Describe optimal outcomes for programs by evaluating their maturity across key operational areas.
\end{itemize}

In this report, the GMI is applied to four major Ethereum Layer 2 grant programs -- \textit{Arbitrum}, \textit{Mantle}, \textit{Taiko Labs}, and \textit{Optimism}. These case studies highlight areas where Web3 grant programs require improvement, particularly in \textit{standardizing processes}, enhancing \textit{transparency}, and increasing \textit{community participation}.
\end{abstract}

\begin{IEEEkeywords}
maturity model, Web3 governance, decentralised autonomous organisations, crypto-economic systems, mixed-methods
\end{IEEEkeywords}

\section{Introduction}\label{sec_1}
Web3 grants are a relatively new phenomenon that has seen little standardisation. Thus, grant programs in Web3 lack a systematic framework for comparing and evaluating their outcomes. While some work has been done on Web3 grants in the context of creating a framework for decentralised science (DeSci)~\cite{ding_desci_2022} and grants found mention in research on Web3 governance~\cite{allen_exchange_2022}, no further reference to Web3 grants exists in the literature across all disciplines, including studies on procurement. The lack of exploratory and systematic literature describing the phenomenon of Web3 grants poses significant challenges to Web3 grantors, grant operators, grant-giving decentralised autonomous organisations (DAOs), and grantees because all actors rely solely on industry knowledge and informal guides.

In absence of a theoretical framework, the structure and benefits of decentralised grant programs cannot be reliably measured or tracked. This poses a problem for evaluating the utility of programs for allocating funding through distributed systems, specifically for using blockchain to disperse grants. Decentralised networks are said to be more open and less rigid than hierarchically and centralised structures~\cite{ding_desci_2022}, but their inaccessibility to outsiders creates information asymmetries both for applicants and funders. If no systematic information is available on grant programs that are supposed to be open, Web3 grant processes become more closed to non-expert audiences and cast doubts over the equity of the funding allocation.

Moreover, the effects of grants even in established sectors are difficult to measure, such as in the case of research grant funding by the European Union~\cite{selebaj_effects_2021}. Both Web3 grant programs and EU-backed grants are positioned to address a diverse audience. Meanwhile, limited information and the heavy use of jargon in Web3 grants undermine the objective of diversity. In this case, grant operators face the challenge of properly assessing funding requests and their impact that is skewed by applicant self-selection. Therefore, Web3 grant programs struggle to ascertain whether the population of applicants represents the desired range of fields or is a second order effect of the limitations in the design of the program. Even for long-standing grant initiatives, like those by the EU, measuring grant program effects is difficult and improved only recently~\cite{selebaj_effects_2021}. In the absence of formal, reliable, and complete data for Web3 grant programs, formal models for measuring capital efficiency cannot be applied~\cite{ odewole_capital_2020}.

To that end, this research proposes the grant maturity framework (GMF), which aims to fill the gap of a systematic framework that provides practitioners with ``a baseline and benchmark for [Web3 grant program] maturity and identifying areas for improvement''~ \cite{dener_govtech_2021}. The GMF builds on prior research by the World Bank Group’s (WBG) towards a government technology (GovTech) maturity index (GTMI). As for the GTMI, the GMF does not aim to rank individual grant programs. Rather the GMF introduces an exploratory weighted composite framework, which is constructed using mixed methods. For describing Web3 grant programs, the research is organised through three research questions.

\begin{enumerate}
\item How can the structure and outcomes of Web3 grant programs be measured?
\item What maturity levels do popular Web3 grant programs exhibit?
\item Which lessons can be learned for the technical design of Web3 grant platforms?
\end{enumerate}

The first research question aims at providing overview of the objectives, processes, and organisational structures that underpin Web3 grant programs. For responding to the second question, the maturity framework is applied to four concrete Web3 grant programs for assessing their maturity. Towards answering the third research question, this paper analyses the outcomes of implementing the results from the GMF within the scope of an EU next generation internet (NGI) project that develops a grant funding platform according to the principles of a user-centric approach (UCA). This research problem is addressed by using mixed-methods for comparing the grant programs of four popular Ethereum layer-two (L2) ecosystems that have a token. Accordingly, the maturity is assed of the grant programs run by Optimism, Arbitrum, Mantle (formerly known as BitDAO), and Taiko. The GMF analyses the effectivity of organisational measures towards improving Web3 grant program operations and outcomes through round-over-round indicators and expert assessments. As a result, the GMF serves as a toolkit for Web3 grant operators, program principals, and stakeholders for tracking the effects of Web3 grant programs and translating learnings into actionable insights for the development of grant funding platforms.

\section{Background}\label{sec_2}

This section provides an overview of the state of grant research in management science and Web3-specific literature. Grounded in the literature on public innovation funding and management science~\cite{albors_impact_2011,bartle_review_2003}, Web3 grants are understood as capital allocation using blockchain technology and funding its development. Although representatives of Web3 may conjecture that Web3 cannot be described by applying traditional organisational theory, the emergence of novel grant mechanisms in Web3, such as \textit{quadratic funding}, illustrates the organisational shift from hierarchical structures to disintermediated crypto-economic systems Web3 stands for~\cite[p.~501]{shermin_disrupting_2017}. In general, grant programs also mitigate investment risk by reducing investment ticket sizes and funding projects to gain traction. The cryptocurrency and Web3 sector is well known for heightened investment risks, thus, grants can take a key role in reducing the risk for capital allocators by providing funding incrementally and connecting it to non-financial support, such as technical assistance~\cite[p.~6]{gilbert_sustainable_2019}.

As a concept, grants are interesting because they fuse organisational concepts of Web3, such as decentralised autonomous organisations (DAOs) with conventional economic studies of the public and private sector~\cite{ding_desci_2022,monteiro_decentralised_2023,wang_self-sovereign_2020}. Following, DAOs are defined as “complex [...] smart contract [...] governance”~\cite[p.~501]{shermin_disrupting_2017} allowing for “human–machine hybrid [...] self-organization with no centralized hierarchy”~\cite[p.~1564]{ding_desci_2022}. In practice, the first DAO introduced the notion of DAOs acting as a “democratic investment fund”~\cite[p.~4]{santos_dao_2018}. According to this definition, DAOs combine established financial logic with disruptive technology and novel organisational structures. 

Given the importance that is attributed to DAOs in respect of Web3 market structure, these organisations are now challenging the hegemonic structure of economic actors, most importantly the \textit{enterprise}~\cite{wang_novel_2024}. Stringently, DAOs may also take over the role of a procurement body or intermediary for innovation procurement in Web3. Innovation procurement intermediation is defined as “provid[ing] a link between at least two entities which need to connect in order to generate or adopt innovation”~\cite[p.~416]{edler_connecting_2016}. In this regard, this paper argues that it is possible to transfer knowledge from conventional grant programs to Web3 grants because any grant program must engage with its stakeholders to receive legitimacy and instil trust in its intermediation. To this end, both conventional and Web3 grant programs create blog posts, hold webinars, publish social media posts, and disseminate educational materials. Thus, even Web3 grant programs cannot exclusively rely on blockchain rails, but at a minimum must communicate with the funding body and the grant recipients. This communicative role is fulfilled through conventional means rather than on-chain. As this contribution attempts to introduce a general scoring methodology for Web3 grant programs, it now proceeds to define the key concepts of the framework, \textit{i.e.} \textit{grants}, \textit{Web3}, and \textit{maturity}.

Grants are broadly defined as ``a financial donation awarded by the contracting authority to the grant beneficiary'', which can be tied to a specific action or unrestricted for objectives of a specific organisation~\cite{european_commission_grants_2023}. Operators and observers of government grants for research and development (R\&D) have been analysing, evaluating, and measuring their impact almost as long as the concept exists, leading to criticism towards the effectiveness of grants in these contexts~\cite{howell_financing_2017,lerner_government_2000}. It was found that in the case of the United States (US) Small Business Innovation Research (SIBR) government R\&D grants improved employment and monetisation capacities of recipients, however, did not lead to an increased likelihood of venture capital investments~\cite{lerner_government_2000}.

Thus, concerns over grant efficiency are not limited to Web3, but apply to many grant programs, both public and private. If funding efficiency does not distinguish Web3 grants from conventional grant programs, what makes Web3 grants different from public innovation procurement? While there is over forty years of evidence on the efficiency of the organisational structure for traditional innovation funding allocation~\cite[p.~4]{holmstrom_agency_1989}, organisational structures in Web3 are more volatile than those outside of crypto-economic systems~\cite[p.~25]{zuo_development_2023}. Web3 only emerged around 2014, when the technology was available to ``embrace[...] a set of protocols based on blockchain, which intends to reinvent how to return data ownership to users and let everyone equally participate in it''~\cite[p.~4]{wan_web3_2023}. Given that definition, the key difference between Web3 and conventional grants is the expected maturity of the programs because Web3 is a relatively new phenomenon.

\subsection{Web3 Grant Programs}\label{sec_2.1}

Although challenges for quantitatively measuring output and impact~\cite{ding_desci_2022,howell_financing_2017} are documented for grant programs in- and outside of Web3, the novelty of Web3 exacerbates these challenges. Based on the findings in~\cite{ding_desci_2022,wan_web3_2023} DAOs are an important organisational design in Web3 and a dominant feature of Web3 grant programs. The reliance of DAOs on smart contracts and decentralised technologies renders this form of organisation also as a specific governance technology, DAO members are subject to. Therefore, this subsection specifies the context of the GMF as modelling the \textit{maturity} of governance technologies in the context of Web3 grant programs and draws from research by the World Bank Group (WBG).

Exploratory and inductive research on Web3 grants provides a broad overview of the Web3 grant landscape and common challenges, for example, outcome measurement, but does not follow a systematic approach~\cite{leventhal_state_2023_long,leventhal_state_2024}. The World Bank GovTech Maturity Index (GTMI) provides structure for mapping the relationship between resource governance and innovation output of Web3 grant programs. It is an example for how governance mechanisms can be associated with concrete technology outputs. The concept of the GTMI is interesting for analysing and comparing several Web3 grant programs because it does not rank individual program performance. Hence, the GMF incorporates the GTMI framework for measuring the individual performance of participants in the quadrants ``core government systems, public service delivery, citizen engagement, and GovTech enablers''~\cite[p.~5]{dener_govtech_2021}.

These variables can be transposed into the context of Web3 grants, where the GMF comprises grant governance systems, grant delivery, stakeholder engagement, and grant program enablers. The categories are relevant for measuring the maturity of Web3 grant programs as they focus on organisational aspects rather than directly tracking inputs and outputs of the programs for determining their efficiency. In this regard, agents are those, who are delivering the outcomes and impacts, \textit{i.e.} the applicants or grantees. Principals are donors, funders, or even DAOs themselves, given that a DAO owns the funds dedicated to the grant program.

DAOs are known vehicles to deliver grants in Web3~\cite{austgen_dao_2023}, thus it is relevant to address challenges of Web3 grant programs when making use of DAOs as organisational structure for governing the funds. More importantly, it becomes evident that ``grants [...] directly surface the relationship between DAOs and traditional challenges in treasury management and public finance''~\cite[p.~27]{tan_open_2023}. Consequently, the concept of maturity allows the description of the evolution of grant-giving, from classical efficiency considerations~\cite{holmstrom_agency_1989} to organisational challenges posed by grants based on crypto-economic systems.

Web3 grant programs typically encompass a blockchain network operator or backer, a program manager, applicants, and stake-holding communities~\cite{gilbert_sustainable_2019,howell_financing_2017}. From an organisational perspective, Web3-specific programs differ from conventional grant programs in two aspects. Firstly, backers and program managers behind Web3 grant programs typically are positioned in the private sector and their organisational structure does not always conform to established governance structures, such as enterprises and public sector entities. Secondly, Web3 grant programs exist for a significantly shorter period compared to governmental grant programs. Therefore, blockchain and DAOs as ``governance technologies''~\cite{brekke_hacker-engineers_2021,shermin_disrupting_2017}  are a catalyst for organisational experimentation and innovation in Web3 grant program design. The openness and transparency of crypto-economic systems~\cite[p.~54]{santos_dao_2018} position Web3 programs in the grant landscape and drive the outcomes of Web3 grant program.

Web3 program operators can choose from an ever increasing set of governance mechanisms for the selection and distribution of grants, which is seen as a quality marker by Web3 participants~\cite{owocki_gitcoin_2024}. Yet, the growth and adoption of novel allocation mechanisms risks distracting from appropriately communicating and engaging with applicants and grantees. Thus, technological innovation and the organisational role of program managers create conflicts that can negatively affect Web3 grant maturity.

\subsection{Grant Program Maturity}\label{sec_2.2}

Given the limitations of quantitative grant impact assessments, econometric measures may not provide sufficient detail on the effects of associated Web3 grant programs. In other words, for evaluating Web3 grant operators' performance of supporting technological innovation and providing organisational structure for applicants, observing changes of transaction count, transaction velocity, and total value locked (TVL) are not enough. Instead, the GMF measures Web3 grant program \textit{maturity} through a composite mixed-method framework. Thus, this approach adopts the WBG GTMI complementary positioning to traditional macroeconomic indicators for nation states, such as the gross domestic product (GDP), and monetarist metrics such as the quantity theory of money~\cite{sun_understanding_2004}.

In general, maturity models emerged from the intersection of innovation research and economics in the health care sector in regard to digital technologies~\cite{van_ede_assembling_2024,knosp_research_2018}. Grey literature and policy documents have applied and popularised the concept~\cite[see also]{dener_govtech_2021,queensland_audit_office_risk_2023}, but only \cite{kucinska-landwojtowicz_organizational_2023} provides an explicit definition of \textit{maturity}. Based on the state-of-the-art analysis by~\cite{kucinska-landwojtowicz_organizational_2023} and the application of maturity modelling for ``evaluat[ing] the current level of operational development''~\cite{yatskovskaya_integrated_2018},Table~\ref{tab:grant_maturity} puts forth an explicit approach for defining maturity in the context of Web3 grant programs. For this contribution, maturity is understood ``in dynamic terms – as a process [...] [and] as a specific state or degree of perfection'' for measuring an organisation's success''~\cite[p.~62]{kucinska-landwojtowicz_organizational_2023}.

More stringently, for measuring Web3 grant programs structure and outcomes, maturity ``represents an anticipated, desired, or typical evolution path of these objects shaped as discrete stages''~\cite[p.~213]{becker_developing_2009}. The GMF Web3 grant maturity model operationalise this definition by classifying programs into four stages. Program properties are described through the definition maturity stage, distinct program features, and relevant improvements for Web3 grant programs in the respective stage. It is implicit that any Web3 grant program must pass through one stage in order to reach the next one, which is grounded in the theory~\cite{yatskovskaya_integrated_2018}. Thus, the GMF incorporates a higher-better logic when scoring Web3 grant programs, which is underpinned by the notion that maturity changes, while programs in a specific stage of maturity exhibit distinct properties for this stage.

\begin{table}[htbp]
\caption{Stages of Maturity for Web3 Grant Programs}
\begin{center}
\footnotesize
\begin{tabular}{p{2cm}p{6cm}}
\hline
\textbf{\textit{Maturity Stage}} & \textbf{\textit{Description}} \\
\hline
Experimental & Focus on exploring new funding mechanisms. Simple structure with limited stakeholder engagement. No formal processes or timelines. \\
\hline
Foundational & Basic governance and program structure defined. Initial mission, vision, and objectives. Simple application process and evaluation criteria. \\
\hline
Developmental & Improved structure with clearer application processes. Greater resource allocation and decision-making beyond core organisation. Focus on impact metrics and community feedback. \\
\hline
Advanced & Standardised processes with dedicated infrastructure and staff. Transparency, regular audits, and community engagement. Impact measurement tools and comprehensive decision-making. \\
\hline
\end{tabular}
\end{center}
\label{tab:grant_maturity}
\end{table}

Flexible and dynamic definitions of maturity have also led to criticism towards the concept~\cite[p.~8]{pereira_review_2020}. To address this shortcoming, the GMF links the conceptual maturity stages in Table~\ref{tab:grant_maturity} to a multivariate composite framework. The maturity model of the GMF, thus is adaptive through using organisational maturity~\cite{andersen_e-government_2006,johansson_roadmap_2019}, while providing replicable results based on the mixed-method operationalisation of an explicit and concise definition of the term.

\section{Methodology}\label{sec_3}

This research follows a mixed-method design to assess Web3 grant programs through the lens of their maturity, triangulate the findings, and suggest actionable improvements in form of Web3 grant tooling. The GMF balances organisational improvements with efficiency gains by aggregating qualitative assessments and quantitative indicators. Outcomes of the framework were used to develop a Web3 grant platform and collect feedback. In the future, the GMF may be used as a tool for Web3 grant program strategy development and audits. The methodology is structured in three phases. Firstly, an expert assessments were collected using a questionnaire called \textit{rubric scoring framework}~\cite{biedermann_evaluating_2024}. This data is was used to refine the quantitative indicators in the GMF, which were used to collect primary data from the grant programs. Secondly, the GMF was calculated. Lastly, the GMF informed the Web3 grant platform development and structured the collection of feedback from practitioners. The second round of qualitative feedback triangulated the findings~\cite{creswell_designing_2017,datta_paradigm_2006}.

Primary data collection was based on known challenges in conventional grant programs, such as capital efficiency, organisational design, and politico-economic stances of grant program operators~\cite{lerner_government_2000}. For underpinning the comprehensive definition of maturity, and Web3 grant maturity specifically, the GMI contributes falsifiable data~\cite[p.~17]{popper_objective_1973} by collecting data and systematising it. This approach verifies whether the concept of \textit{maturity} actually applies to Web3 grant programs~\cite[p.~7]{santos_dao_2018} and holds true~\cite{hutton_abstraction_1990}. Thus, the GMF overcomes the chasm between explorative, yet substantiated research~\cite{mukumbang_retroductive_2023}.

The GMF is based on the ``ontological assumption[s]''~\cite[pp.~94]{mukumbang_retroductive_2023} that Web3 grant programs maturity exist the token utility is a governance tool~\cite{beck_governance_2018,messias_understanding_2024}, which can be used to quantify ``the value of communication'' as measure of maturity~\cite{johansson_roadmap_2019}. This assumption is challenged in the two qualitative data collection phases, which informed the grouping of data according to six rubrics. Thus, the GMF score expresses the maturity of Web3 network subsidies that are given as grants in native blockchain tokens for creating economic and social value. Similar research evaluates the effectiveness of chief executive officer (CEO) grant packages~\cite{billett_stockholder_2010} and calculates network value~\cite{papaioannou_business_2023}.

\subsection{Rubric Scoring Framework and Indicator Clustering}\label{sec_3.1}

Procurement of technology and software is documented in enterprises and the public sector~\cite{bartle_review_2003,johansson_roadmap_2019,uyarra_barriers_2014}, but only sporadically for DAOs~\cite{monteiro_decentralised_2023} without taking maturity into account. Following, the GMF incorporates an evaluative part, where grant programs are scored holistically and based on a discrete set of categories. These scoring categories, were developed inductively and are the basis for the rubric scoring framework, as well as the GMF. Broadly, the GMI consists of the six clusters defined in Table~\ref{tab:grant_rubric} below.


\begin{table}[htbp]
\caption{Web3 Grant Program Rubric Scoring Framework}
\begin{center}
\footnotesize
\begin{tabular}{p{3.8cm}p{4.5cm}}
\hline
\textbf{\textit{Rubric}} & \textbf{\textit{Assessment Criteria}} \\
\hline
Focus Areas, Objectives (FAO) & Defined focus areas and strategic goals \\
\hline
Program Structure (PSO) & Clarity of decision-making processes \\
\hline
Governance (GOV) & Governance structure, transparency, funding sources \\
\hline
Effectiveness, Impact (EFI) & Explicit goals, success criteria, impact assessment \\
\hline
Transparency (TAC) & Transparent, accountable reporting mechanisms \\
\hline
Community Engagement (COM) & Community reach, engagement, incentives \\
\hline
\end{tabular}
\end{center}
\label{tab:grant_rubric}
\end{table}

Based on this rubric scoring framework the scoring form was developed, which was shared among five participants for conducting the Delphi phase of the research, who were representatives of zkSync, Arbitrum, Octant, Balancer, and GitCoin, as well as non-affiliated Web3 grant professionals. The appraisal approach is supported by practitioners in the field, who reported that this is their preferred method of selecting grant programs in their business routine. Using the Delphi method, as suggested for health management science~\cite{van_ede_assembling_2024}, the appraisal scheme requires evaluators to score programs in the categories and subcategories on a continuous scale from one (1) to five (5). A low score represents low maturity, a score of three (3) signals intermediate maturity, and five (5) signifies high perceived maturity. Evaluators were requested to add comments for each score given. The scores of the subcategories were then aggregated to category scores and a written justification is given for the overall evaluation.

Qualitative results were then used to select available quantitative data points for the set of grant programs to be included as indicators in the GMF. Meanwhile, quantitative scores were included also included as indicators for each rubric in the GMF. This approach emphasises the importance of organisational maturity, by explicitly incorporating practitioners perspectives and enhancing transparency among stakeholders of Web3 grant programs. This can lead to increased perceived legitimacy among expert audiences~\cite[p.~116]{curtin_does_2006}, but also support Web3 grant processes in actively building rapport~\cite{suddaby_legitimacy_2017}.

\subsection{Construction of the Grant Maturity Framework}\label{sec_3.2}

The GMF consists of 46 data points that are grouped in the six rubrics clusters (Table~\ref{tab:grant_rubric}). A subset of 40 data points are included in the index with a non-zero weight, the remaining six (6) were taken into account to provide a ``rich description''~\cite[p.~197]{mcbride_sailing_2019} of selected Web3 grant programs in the dataset. Accordingly, the GMF and its rubric scores are micro-level composite maturity indices. Due to the nascence of Web3 grant research and the volatility of Web3 at large, the study focused on compiling a comprehensive list of relevant characteristics for assessing the maturity of a grant program. Therefore, the GMF is best understood as an authoritative and exhaustive collection of indicators that are capable of describing the maturity of a selected group of Web3 grant programs.

The indicator composition was informed by the results of Delphi study. This has led to several exclusions of indicators that are limited to a single program and rely too heavily on binary scores, are jurisdiction-specific, and differ depending on the grant size, such as ``presence of due diligence processes'', ``awarding unrestricted grants'', and ``a public retrospective exists''. Thus, the GMF is a baseline maturity model for Web3 grant programs for grasping the overall stance and structure of a given grant program, which follows the recommendations by~\cite{oecd_handbook_2008} on the construction of multivariate composite indicators.

The composite index consists of two main components. Namely, the rubric scores for all analysed Web3 grant programs are given as normalised maturity composites and the GMF is provided as a normalised composite of all rubric scores for each program. The framework follows an additive logic and assumes equal weights both for the indicators of the rubric scores, but also for the rubric scores themselves. For the calculating the GMF maturity score, first, the data was collected and listed according to the structure in Table~\ref{tab:grant_rubric}. Second, the data was normalised using the min-max normalisation function. For normalising the indicators in the rubrics, as well as the rubric scores, min-max normalisation method was used to ensure comparability across variables and construct validity of the composites, i.e. “[a]void adding up apples and oranges”~\cite[p.~27]{oecd_handbook_2008}. It was chosen because it is an accepted method for normalisation in studies that deal with the governance of uncertainty~\cite{winters_when_2004}. Third, the rubric score composites were calculated and aggregated. Fourth, the rubric scores were normalised and aggregated for calculating the final GMF composite. Based on this calculation, the maturity stages in Table~\ref{tab:grant_maturity} were defined as quartiles from zero to one.\\

\subsubsection{Composite Rubric Score Calculation}\label{sec_3.2.1}

The composite rubric score for indicators of one rubric for each program was calculated using the formula
\footnotesize
\[
CRS_{ik} = \sum_{j=1}^{n} w_{jk} X'_{ijk}
\]

where:

\begin{itemize}
    \item \( CRS_{ik} \) is the composite indicator for the \( k \)-th rubric score of the \( i \)-th program,
    \item \( X'_{ijk} \) is the normalised value of the \( j \)-th variable for the \( k \)-th rubric score of the \( i \)-th program,
    \item \( w_{jk} \) is the weight assigned to the \( j \)-th variable for the \( k \)-th rubric score of the \( i \)-th program, implying equal weights, \( w_{jk} = \frac{1}{n} \),
    \item \( n \) is the number of variables for a given rubric score of a given program.
\end{itemize}\vspace{7pt}
\normalsize
\subsubsection{Composite Grant Maturity Framework Score}\label{sec_3.2.2}
The composite GMF score for rubric scores was calculated using the formula
\footnotesize
\[
GMF_i = \sum_{k=1}^{m} w_k CRS'_{ik}
\]

where:

\begin{itemize}
    \item \( GMF_i \) is the composite indicator for the \( i \)-th program,
    \item \( CRS'_{ik} \) is the normalised value for the \( k \)-th rubric score of the \( i \)-th program,
    \item \( w_k \) is the weight assigned to the \( k \)-th rubric score of the \( i \)-th program, implying equal weights, \( w_k = \frac{1}{m} \),
    \item \( m \) is the number of rubric scores for a given program.
\end{itemize}\vspace{7pt}
\normalsize
\subsubsection{Construct Validity and Reliability}\label{sec_3.2.3}
Although this research is exploratory, it is important to consider the construct validity of the Web3 grant maturity framework. Indicators were grouped and nested into rubrics, decoupling and abstracting the overall framework from the underlying indicators and construction of the composites is replicable. Yet, currently the index assumes that all rubrics are equally important. Further research may relativise this stance and introduce rubric weights with \( 0 \leq w_{ij} \leq 1 \), which would also allow the presentation of program scores on a \( 0 \leq w \leq 1 \) scale, rather than listing their absolute composites.

\subsection{Use Case-Driven Stakeholder Feedback}\label{sec_3.3}
After the GMF for the four Web3 grants programs was calculated, a the research team developed prototype for Web3 grants management platform to test the findings following a user-centric design approach. This meant iteratively refining the functionalities identified as most relevant to Web3 grant program maturity based on direct stakeholder feedback. The methodology was grounded in qualitative user research, incorporating semi-structured user interviews and participatory workshops with Web3 grant funders and applicants. A qualitative approach was chosen over a quantitative, anonymised data collection method as it allows for a deeper exploration of users' underlying emotions, perceptions, and motivations. This approach enabled studying people in depth, as suggested by~\cite{norman_design_2013}, helping not only to understand what users expect from a Web3 grant management platform but also to uncover why and how they believe those expectations of maturity can be best met based on their experience and expertise.

The Design Thinking framework was an important component of the development process, following five iterative stages: empathising with users, defining their needs, ideating solutions, prototyping, and testing~\cite{ambrose_basics_2010,wolniak_design_2017}. Early user testing focused on low-fidelity wireframes demonstrating the creation of grant programs and application submission processes. Continuous engagement with the target audience in validation environments continued at a Web3 hacker house, where the first clickable prototype was developed, as well as through design thinking workshops, where the researchers engaged with Web3 program operators and applicants in person.

\section{Results}\label{sec_4}

This section presents the results of the analysis conducted on popular Web3 grant programs using the Grant Maturity Index (GMI). The focus is on examining the maturity levels of key programs, including the largest programs of Arbitrum (ARB), Optimism, Mantle, and Taiko, and evaluating their structures, processes, and outcomes. The results also aim to answer the research questions posed at the outset of the study: how the structure and outcomes of Web3 grant programs can be measured, what maturity levels popular Web3 grant programs exhibit, and which lessons can be learned for the technical design of Web3 grant platforms. These questions are addressed by reviewing the rubric scores, identifying patterns across the programs, and drawing insights from both quantitative and qualitative data collected.

\subsection{Measuring Web3 Grant Program Maturity}\label{sec_4.1}

The structure and outcomes of Web3 grant programs can be measured using the GMF, which incorporates six key rubric categories displayed in Table~\ref{tab:grant_rubric}. Each rubric category includes multiple indicators that evaluate aspects of the program, such as grant size, evaluation criteria, organisational structure, governance mechanisms, and community involvement. The indicators in the FAO rubric assess grant types, funding methods, and evaluation timeframes, among other factors. The PSO rubric evaluates the organisational structure of the grantor and the allocation of funds. The GOV rubric includes questions on the program's mission and vision documents and its alignment with objectives like network growth or philanthropy. The EFI rubric measures how effectively the program achieves its stated goals, while the TAC rubric focuses on transparency and accountability in decision-making processes. Finally, the COM rubric looks at the level of community engagement, including the applicant count, average grant size, and program management team size. More concretely, the rubrics included in the GMF cover the following areas:\\

\footnotesize
\begin{itemize}
    \item \textbf{FAO} evaluates the goals and types of grants offered
        \subitem Grant size in total, minimum, and maximum
        \subitem Evaluation timeframe
        \subitem Funding types, \textit{e.g.} lump sum or milestone-based
    \item \textbf{PSO} assesses organisational and funding structures
        \subitem Origin of funds, \textit{e.g.} token launch, donors
        \subitem Fund allocation and governance structures
    \item \textbf{GOV} examines alignment between objectives and governance
        \subitem Documentation
        \subitem Alignment with program objectives
    \item \textbf{EFI} measures success in goal attainment
        \subitem Evaluation criteria and reporting
        \subitem Domicile of associated legal entities
    \item \textbf{TAC} evaluates transparency and accountability mechanisms
        \subitem Grant accessibility and accountability
    \item \textbf{COM} assesses community involvement
        \subitem Applicant count and grants allocated per round
        \subitem Community engagement in governance
\end{itemize}\vspace{7pt}
\normalsize

By examining these rubrics and their respective indicators, the GMF allows for a structured and comprehensive assessment of Web3 grant programs. In aggregate these indicators offer insights into the maturity of each program by informing the quantitative categorisation of Web3 grant programs into stages of maturity, which were refined through the Delphi study and stakeholder feedback. The quantitative measurement was introduced inductively after comparing the aggregate scores of grant programs with the appraisal comments given by expert reviewers, which led to a naive quartile-based stratification of grant programs according to maturity in Table~\ref{tab:maturity_stages}.

\begin{table}[htbp]
\caption{Refined Maturity Stages}
\begin{center}
\footnotesize
\begin{tabular}{p{1.5cm}p{2.5cm}p{3.5cm}}
\hline
\textbf{\textit{Maturity}} & \textbf{\textit{Quartile Range}} & \textbf{\textit{Description}} \\
\hline
Experimental & \( 0 \leq GMF < 0.25 \) & Exploratory with limited structure and governance \\
Foundational & \( 0.25 \leq GMF < 0.5 \) & Programs start to define objectives, structures, and governances \\
Developmental & \( 0.5 \leq GMF < 0.75 \) & Clear structures are in place with defined processes for allocation \\
Advanced & \( 0.75 \leq GMF \leq 1.0 \) & Programs have robust and transparent governance with impact measurement \\
\hline
\end{tabular}
\label{tab:maturity_stages}
\end{center}
\end{table}

\subsection{Maturity Levels of Popular Web3 Grant Programs}\label{sec_4.2}

The following grant programs were selected for the study: the Arbitrum Short Term Incentive Program (STIP), Arbitrum STIP Backfund, Arbitrum STIP Bridge, Arbitrum Long-Term Incentive Pilot Program (LTIPP), the Mantle Grants Program, the Optimism Mission Rounds, and Taiko's Incentivisation Grant Program. These programs were chosen as they represent a cross-section of Web3 ecosystems and cover a variety of maturity levels, from experimental to advanced stages. The maturity levels of these programs are detailed in Table \ref{tab:gmf_composite_score}, which provides both additive and normalised composite scores.

\begin{table}[htbp]
\caption{GMF Composite Scores}
\begin{center}
\footnotesize
\begin{tabular}{p{2.7cm}p{2cm}p{2.3cm}}
\hline
\textbf{\textit{Grant Program}} & \textbf{\textit{Additive} \( GMF\)} & \textbf{\textit{Normalised} \( GMF \)} \\
\hline
ARB STIP \& Backfund & 3.6653 & 0.4349 \\
ARB STIP Bridge & 3.7582 & 0.5251 \\
ARB LTIPP & 4.2679 & 0.6755 \\
Mantle & 3.1787 & 0.2729 \\
Taiko & 3.1191 & 0.2334 \\
Optimism & 3.6397 & 0.6105 \\
\hline
\end{tabular}
\end{center}
\label{tab:gmf_composite_score}
\end{table}

These results show that the ARB LTIPP program achieved the highest score with a normalised composite of 67.55\%, placing it in the \textit{developmental} stage, which corresponds to the third quartile. The Optimism Mission Grants, with a normalised composite score of 61.05\%, also fall within the \textit{developmental} stage but with slightly lower maturity than ARB LTIPP. In contrast, the Taiko and Mantle programs scored lower, with normalised composites of 23.34\% and 27.29\% respectively, which places them in the \textit{experimental} stage, indicative of their early-phase status and need for further development.

The detailed rubric scores for each program are presented in Table~\ref{tab:rubric_scores}. This table shows the normalised scores for each program across six key rubric categories: Focus Areas and Objectives (FAO), Program Structure and Organisation (PSO), Governance (GOV), Effectiveness and Impact (EFI), Transparency and Accountability (TAC), and Community Engagement (COM). Each category is scored on a scale from 0 to 1, with higher scores indicating better performance in that particular aspect of the program.

\begin{table}[htbp]
\caption{Normalised Rubric Scores of Analysed Programs}
\begin{center}
\footnotesize
\begin{tabular}{p{2.9cm}p{0.5cm}p{0.5cm}p{0.5cm}p{0.5cm}p{0.5cm}p{0.5cm}p{0.5cm}}
\hline
\textbf{\textit{Grant Program}} & \textbf{\textit{FAO}} & \textbf{\textit{PSO}} & \textbf{\textit{GOV}} & \textbf{EFI} & \textbf{\textit{TAC}} & \textbf{\textit{COM}} \\
\hline
ARB STIP \& Backfund & 0.6856 & 1.0000 & 1.0000 & 0.0769 & 0.5455 & 0.2207 \\
ARB STIP Bridge & 1.0000 & 0.9184 & 0.9932 & 0.0000 & 0.5455 & 0.0000 \\
ARB LTIPP & 0.1987 & 0.9892 & 0.9932 & 0.6923 & 0.9091 & 1.0000 \\
Mantle & 0.1477 & 0.6494 & 0.4694 & 0.6154 & 0.0000 & 0.3834 \\
Taiko & 0.0000 & 0.7347 & 0.0000 & 0.0000 & 0.0000 & 0.2040 \\
Optimism & 0.2210 & 1.0000 & 0.0000 & 1.0000 & 0.0000 & 0.4419 \\
\hline
\end{tabular}
\end{center}
\label{tab:rubric_scores}
\end{table}


For example, Arbitrum's STIP \& STIP Backfund program performed particularly well in the Governance (GOV) category, scoring a perfect 1.0000, indicating a well-defined governance structure. However, it scored lower in Effectiveness and Impact (EFI) with 0.0769, reflecting a need for stronger measures of success and impact. Similarly, the Optimism program excelled in Effectiveness and Impact (EFI) with a perfect score of 1.0000 but scored lower in Community Engagement (COM) with 0.4419, suggesting room for improvement in its community outreach and involvement efforts.

In contrast, the Taiko program shows lower scores across most categories, with particularly weak scores in Focus Areas and Objectives (FAO) and Governance (GOV), indicating it is still in the experimental phase, with much work to be done in terms of defining clear objectives and governance structures. Mantle also shows lower scores, particularly in Governance and Transparency and Accountability (TAC), indicating that it is still in the foundational phase and requires more robust processes and structures.


\subsubsection{Optimism Growth and Experiment Grants}\label{sec_4.2.1}
The Optimism Mission Grants program, part of the larger Optimism grant initiative, scored a normalised composite of 61.05\%, placing it in the \textit{developmental} stage. The program’s objective is to drive network adoption and support applications that align with the goals of the Optimism Collective. This program has matured beyond its initial exploratory phase, with well-established evaluation criteria and some degree of transparency in its decision-making process. However, there is still potential to expand its impact by diversifying supported projects and improving its governance framework.

\subsubsection{Arbitrum Grant Programs}\label{sec_4.2.2}
Arbitrum’s multiple grant programs, including the STIP, STIP Backfund, STIP Bridge, and LTIPP, represent a range of maturity levels. The STIP and STIP Backfund programs scored normalised composites of 43.49\% and 52.51\%, respectively, indicating their position in the \textit{foundational} stage. These programs are designed to provide short-term incentives for network growth, but they lack some of the more advanced structures and processes seen in the LTIPP program.

The Arbitrum LTIPP program, with a normalised composite of 67.55\%, is in the \textit{developmental} stage, indicating its more advanced maturity. The LTIPP program focuses on long-term incentives and has a more developed governance structure, well-defined objectives, and measurable impacts.

\subsubsection{Taiko Labs Grants}\label{sec_4.2.3}
The Taiko Labs Incentivisation Grant Program scored a normalised composite of 23.34\%, placing it in the \textit{experimental} stage. While the program has ambitious goals for scalability and network adoption, it is still in the early stages of implementation. The program has made progress in defining its objectives but is still lacking in some areas such as governance, transparency, and community engagement.

\subsubsection{Mantle Grants}\label{sec_4.2.4}
The Mantle Grants Program, operated by BitDAO, scored a normalised composite of 27.29\%, placing it in the \textit{foundational} stage. While the program has a clear structure, it is still developing its governance processes and transparency mechanisms. To move into the \textit{developmental} stage, the program will need to refine its evaluation processes, improve the visibility of its funding decisions, and enhance its engagement with the broader community.


\subsection{Technology-Driven Improvement of Web3 Grant Maturity}\label{sec_4.3}




\section{Discussion}\label{sec_5}











\section{Conclusions and Future Work}\label{sec_6}


Hence, the GMI aims to not only contribute on a practical level to the understanding of Web3 grant programs, but also lays the foundation for future theoretical work that could challenge established notions of legitimacy. This study is designed, however, to ascertain the characteristics, differences, and applications of Web3 specific grant programs. In turn, the GMI may accelerate the maturing of Web3 grant programs by providing Web3 grant operators a tool for self-assessment and benchmarking. As a result, the GMI is anticipated to increase robustness and integrity of Web3 grants by providing transparency as an incentive for good governance of grant funds and Web3 grant decision making processes. For this reason, the next sections explain the composition, computation, and application of the GMI to be used as a tool for future research.

\begin{table}[htbp]
\caption{Table Type Styles}
\begin{center}
\begin{tabular}{|c|c|c|c|}
\hline
\textbf{Table}&\multicolumn{3}{|c|}{\textbf{Table Column Head}} \\
\cline{2-4} 
\textbf{Head} & \textbf{\textit{Table column subhead}}& \textbf{\textit{Subhead}}& \textbf{\textit{Subhead}} \\
\hline
copy& More table copy$^{\mathrm{a}}$& &  \\
\hline
\multicolumn{4}{l}{$^{\mathrm{a}}$Sample of a Table footnote.}
\end{tabular}
\label{tab1}
\end{center}
\end{table}

\begin{figure}[htbp]
\centerline{\includegraphics[scale=0.1]{suibond.png}}
\caption{This shows the suibond application.}
\label{fig}
\end{figure}


\section*{Acknowledgment}

We thank Eugene and Matthew for review and contributions.

\bibliographystyle{IEEEtran}
\bibliography{IEEEabrv,ieee_icbc_tc_bib}



\end{document}
